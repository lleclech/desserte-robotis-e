\section*{\'Etude de cas}
\addcontentsline{toc}{section}{\'Etude de cas}
Le vendredi 24 janvier, notre équipe a été invitée à un cocktail au sein de l’Union Régional des Ingénieurs et Scientifiques d’Aquitaine organisé par l’association Ricard. Durant ce cocktail nous avons pu faire une étude du terrain et poser des questions aux serveurs.

La soirée débuta vers 19h, il y avait 90 personnes pour une superficie avoisinant les $80m^2$, deux buffet garnit de différents alcools et 3 serveurs. Deux étaient en salle et le dernier serveur faisait la liaison entre la cuisine et la salle. Les mets étaient préparés et chauffés avant le début du cocktail.

En ce début de soirée, le niveau sonore se trouvait au environ de 65 décibels. Les pics étaient à 75 décibels et le niveau minimum à 64 décibels. Les serveurs servaient l’alcool. Les groupes se formaient entre les connaissances.

Les serveurs étaient courtois et pouvaient discuter. Il ne rentrait pas dans les groupes, c’était les groupes qui s’ouvraient à eux. Les personnes les plus enfoncés dans la salle se rapprochaient parfois pour être servi. Il ne s’approchait pas de groupe lorsque ces personnes semblaient avoir une discussion trop sérieuse.

Après le service de l’alcool, il y eut les mets froids, pendant que les buffets ne servaient que de l’alcool. Il était possible de trouver deux poubelles sur les buffets.

Ensuite vient les mets chauds. Les serveurs se déplaçaient dans la salle. Pour s’introduire plus profondément dans la salle, il levait leurs plateaux au-dessus des gens et  indiquait leurs passages par des gestes subtiles. Par exemple, il pouvait toucher discrètement le dos d’une personne avec le dos de leurs mains pour indiquer à la personne de la laisser passer. L’acte était assez bien réaliser pour que la personne ne remarque pas qu’elle se bougeait pour le serveur et pour qu’elle puisse continuer sa conversation sans être dérangé par le serveur.

Après les plats chauds, la salle se vida un peu. Nous nous retrouvions à 76 personnes. Les groupes se dissociaient pour permettre les discussions entre les gens qu’ils n’ont pas pu voir durant la soirée. Le bruit restait à la même intensité. Il y avait moins de monde mais surement que l’alcool faisait monter le ton.

C’est vers 21h que le service des pâtisseries arriva. Après ce service, la salle se vida et les serveurs commencèrent à nettoyer et les gens restant, environs 50, aidèrent à ranger la salle.

Ensuite nous avons pu discuter avec les serveurs sur le maintien du plateau, la manière dont il se déplacé dans le cocktail et comment le cocktail était préparé. 
