 
\chapter{Objectif}

La création de la desserte robotisée a pour but de créer un compromis au serveur de cocktail. Notre but est de le rendre non pas le plus humain possible mais le plus serviable et utile. C’est pourquoi il n’est pas un serveur à part mais un objet mobile avec une intelligence lui permettant de s’occuper des besoins des convives, un robot. Il ne doit donc pas s’imposer à l’atmosphère du cocktail mais bien en faire partie.

\section{Servir les invités}

Le robot a pour objectif principal de permettre le service d’apéritifs aux convives. Il doit donc se déplacer entre les invités et leur apporter les amuse-bouches et le champagne.
//
Lors d’un cocktail, les gens se réunissent autour de groupe de personne qu’il connaissait ou qu’il rencontre, il y a donc des formations de groupes. Le robot doit les repérer et réaliser son service. Pour réussir cet objectif, il devra suivre des conventions de service. Comme le serveur qui apporte les mets sans s’introduire dans le cocktail, le robot doit être capable de présenter les apéritifs et savoir lorsque les personnes dont ils s’occupent sont servies. 
//
Lorsque le service et réalisé, si un hôte peut lui demander de venir le servir. Si ce n’est pas le cas, il retourne à sa déambulation.


\section{être un lien de communication}
