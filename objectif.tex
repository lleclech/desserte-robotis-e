 
\chapter{Objectif}

La création de la desserte robotisée a pour but de créer un compromis au serveur de cocktail. Notre but est de le rendre non pas le plus humain possible mais le plus serviable et utile. C’est pourquoi il n’est pas un serveur à part entière mais un objet mobile avec une intelligence lui permettant de s’occuper des besoins des convives, un robot. Il ne doit donc pas s’imposer à l’atmosphère du cocktail mais bien en faire partie.

\section{Servir les invités}
%"'comprendre"' le client

Le robot a pour objectif principal de permettre le service d’apéritifs aux convives. Il doit donc se déplacer entre les invités et leur apporter les amuse-bouches et le champagne.C’est pourquoi, avant même le service, il est capable de savoir ce qu’il doit avoir sur son plateau. Il ne peut pas savoir ce que ses clients veulent mais il sait ce que le cocktail a promis à ses hôtes. C’est pourquoi il a un planning de son service dans le but de satisfaire les invités.

Lors d’un cocktail, les gens se réunissent autour de groupe de personne qu’il connaissait ou qu’il rencontre, il y a donc des formations de groupes. Le robot doit les repérer et réaliser son service. Pour réussir cet objectif, il devra suivre des conventions de service. Comme le serveur qui apporte les mets sans s’introduire dans le cocktail, le robot doit être capable de présenter les apéritifs et savoir lorsque les personnes dont ils s’occupent sont servies. 

Lorsque le service et réalisé, si un hôte peut lui demander de venir le servir. Si ce n’est pas le cas, il retourne à sa déambulation.

Pour assurer avec brillance son service, il doit permettre d’être un atout à la réussite du contexte social du cocktail. C’est-à-dire qu’il doit être un gain de temps pour ses convives pour qu’ils puissent réaliser leurs objectifs tel que les rencontres où les retrouvailles. C’est pour cela qu’il a assez d’éléments sur son plateau pour satisfaire le plus de personne ; qu’il se déplace de tel sorte que le plus de personnes soient servis et, qu’à travers le robot, les clients trouvent toutes les informations nécessaires sur le cocktail. Ainsi le robot aura réussi à valoriser toutes les personnes se trouvant dans le cocktail.

%désirs de performance -> Avoir tout sur le plateau, Gagner du temps, avoir des infos

\section{\^Etre un lien de communication}

En plus d’être un élément de service, le robot est intrinsèquement un élément de communication. Sans agir sur les relations qu’auront les personnes entre elles, il doit pouvoir s’adapter aux mouvements de formation de groupes et avoir la possibilité d’en être le créateur.

Lors du début d’un cocktail, les groupes ne sont pas formés. Au cours de l’évènement, les gens se rassemblent et au fur à mesures que le cocktail se déroule seuls quelques électrons libres continue de passer de groupes en groupes. On a donc en début de cocktail des gens très dispersés. Et à la fin du cocktail on retrouve un ensemble de groupe.

Pour rendre le robot utile lors de se développement du cocktail, il faut que le robot   est la capacité de réunir les gens. Il est un élément d’échange entre les convives.


%adaptation du design

%Etre un éléments d'échange entre les convives
%Mettre en relation les convives
%Permettre à deux convives d'échanger discrètement
%Se faire accepter par l'humain 
