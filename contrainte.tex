\chapter{Contraintes et Propriété}

\section{Les contraintes anthroposociales}

Les contraintes anthroposociales imposées à la conception d'une
desserte robotisée destinée aux cocktails peuvent être séparées en
plusieurs catégories :
\begin{description}
\item[sensorielles :]
  \begin{enumerate}
  \item Le robot ne doit pas par sa présences déranger le bon déroulement des conversations entre convives.
  \item L'apparence du robot doit être en accord avec l'ambiance et le code vestimentaire du cocktail.
  \item La promité du robot ne doit pas induire de sensations désagréables (bruit, odeur, etc.).
  \end{enumerate}
\item[confiances :]
  \begin{enumerate}
  \item Le robot doit respecter les distances nécéssaires afin de ne pas pénétrer l'espace personnel des convives.
  \item Le robot doit être compréhensible et prévisible afin de ne pas effrayer les convives et de renforcer le sentiment de sécurité à sa périphérie.
  \item Le gabarit du robot doit être soigneusement réfléchie afin de ne pas paraitre comme le dominant de l'interaction aussi bien que pour empêcher une apparence menacante.
  \end{enumerate}
\end{description}

%espace perso
%gène du robot (cmt se faire accepter par l'humain) #AvoirUneEmpriseCorporelSurLeRobot
%Confiance (et non pas sécurité)
%Plaisant

%Petite aide:
%ds le context du cocktail on est vigilant => on est "`fin"' dans nos actions


\section{Les contraintes techniques}

\subsection{Contraintes déduitent du champ social}

\begin{description}
\item[Sensorielles :]
  \begin{enumerate}
  \item Le niveau sonore émit par le robot ne doit pas excéder les 60
    décibels, en effet, il à été mesuré en situation que pour un
    cocktail rassemblant 90 personnes, le niveau sonore variait entre
    65 et 75 décibels, en fonction du temps. On peut d'ailleur noter
    que le volume sonore tend à augmenter malgré une diminution du
    nombre de convive.
  \item La forme du robot devra être affinée le plus possible et
    disposer de carénages amovibles permettant de s'adapter à
    différentes ambiances ainsi que la création de carénages customisé
    par le client, si celui-ci le souhaite.
  \item Afin de ne pas incommoder les convives et leur expérience, le
    robot ne devra pas dégager trop de chaleur ni d'odeur.
  \end{enumerate}
\item[Confiances :]
  \begin{enumerate}
  \item Le robot devra embarquer des capteurs et le software
    nécéssaire au maintient des distances programmées avec les
    convives.
  \item Les mouvements devront être fluide, et un ensembles de
    mouvement plus ou moins évident devront être déterminé et
    implémenter afin de laisser savoir au personnes alentour les
    intentions du robot.
  \item Le robot ne devra pas excéder une tailler de 1m50 et une
    circonférence de 3m, ceci afin de rester dans une position de
    servant par rapport aux convives.
  \end{enumerate}
\end{description}

%bruit
%chaleur
%grandeur

\subsection{Contraintes liée aux services}

\begin{description}
\item[Autonomie :] Le robot doit avoir une autonomie suffisante pour
  tenir toute la durée du service. D'après le livre "Savoir et
  Techniques de restauration - tome 2" ~\cite{Ferret200412} un buffet-cocktail dure en
  moyenne 2h30. Ceci dit, on peut imaginer que le robot sera utilisé
  sur une période plus longue que le cocktail (par exemple pendant la
  préparation ou le rangement) et il faudrait prévoire une marge
  conséquente sur l'autonomie d'autant qu'il est critique que le robot
  ne tombe pas en panne de batterie pendant le service.
\item[Sécurité :] Le robot ne doit pas entrer en contact avec des humains.
\item[Spatiale :] Le robot doit s'adapter aux conventions de densité
  de personnes appliquée dans le milileu de la restauration:
  \begin{itemize}
  \item pour 20 personnes, une surface de $23m^2$ est préconisée.
  \item pour 200 personnes, une surface de $219m^2$ est préconisée.
  \end{itemize}
  On compteras donc une densité d'un peu plus d'un mètre carré par
  personne.
\item[Hygiène :] Le plateau ne doit pas être pollué par le robot.
\item[taille du plateau :] Le plateau fait une taille standard de
  $55*35cm$ s'il est rectangulaire et $40cm$ de diamètre s'il est
  circulaire.
\item[Isolation du plateau :] La temperature du plateau ne doit être
  influencé par le robot (ex: actionneur sous le plateau qui
  chaufferait).
\item[Homogénéité du service :] Le robot doit servir toutes les zones
  géographiques de la salle.
\item[Charge plateau :] Le robot doit pouvoir porter au moins $10Kg$
  pour pouvoir assurer le service et aider les serveurs pendant les
  phases de préparation du cocktail.
\end{description}


%sécurité
%durabilité
%batterie


