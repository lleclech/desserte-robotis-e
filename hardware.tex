\chapter{Spécification techniques}

\section{Le Hardware}

\subsection{Le bras}

Pour assurer les mouvements de service et pour ranger le plateau lors
des déplacement un bras robotisé avec trois degrés de liberté est
suffisant pour reprendre les gestes d'un serveur, il devra cepandant
avoir comme caractéristique la possibilité de se retracter pour ranger
le plateau en position de déplacement.

Le plateau disposera également de capteur à ultrasons et/ou infrarouge
pour permettre de positionner le plateau aux bonnes distances par
rapport aux groupes ou aux personnes, soit un peu moins d'un mètre.
Cela implique cependant que le robot arrive à faire la différence entre 
une personne désirant se servir et une personne n'ayant pas vu le robot 
et ayant une trajectoire dangeureuse.

le plateau comme annoncé dans les contraintes liées au services a
traditionnellement un diamètre de $40cm$ que nous avons décidé de
conserver. Comme il doit être composé d'une matière disposant d'un
bonne isolation thermique, une base en acier inoxydable nous semble
être une solution cohérente.

Un capteur de pression devra être installé sur le plateau pour déterminer 
a quel moment le plateau doit être recharger.

\subsection{Le corps}

\subsubsection{Chassis}
Nous avons choisis de donner au robot un corps cylindrique d'un
diamètre légerement supérieur $45cm$. l'idée étant qu'en position de
déplacement, le plateau s'emboite légèrement dans le corps du robot.
La matière composant l'exterieur du chassis devras être relativement
souple pour minimiser un impact qui n'aurait pas pu être éviter. 

\subsubsection{Base mobile}
Pour ce qui est de la base mobile, nous avons choisi l'utilisation de
roue omnidirectionnelle, car les déplacements du robot demandent une
grande précision du fait de l'évolution en milieu humain, par contre
ce choix impose un sol totalement plat, ce qui exclut d'office tout
service en milieu exterieur.

\subsubsection{Sécurité}
Pour répondre aux contraintes de sécurités, nous nous somme inspirés
du robot numéro 1 (voir le chapitre sur l'état de l'art) et nous avons
choisis de coupler des capteur infrarouges à des capteurs à ultrasons
disposé tout autour du chassis pour permettre au robot d'éviter tout
contact. 

\subsubsection{Interface}
une surface tactile est prévue pour permettre à un utilisateur
d'interagir avec le robot (par exemple pour obtenir des information).

\subsubsection{autonomie}

\subsubsection{Caméra}

Outre les différents capteurs déjà cité pour le plateau ou la sécurité
du robot, une ou plusieurs caméras doit être installée sur le robot pour l'aider
d'une part à repérer des groupes de personnes, et d'autre part une
fois que le robot commence à servir le groupe, à éviter que le robot
aille servir trop de fois la même personne.

\subsubsection{}
