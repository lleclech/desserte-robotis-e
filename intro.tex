\chapter*{Introduction}
\addcontentsline{toc}{chapter}{Introduction}

%surement qu'il faudrait mettre le classique de la réalisation de tte introduction: accroche , problématique, analyse du devellopement du plan

\section{Comment concevoir dans un champ social}
%Ouaouh! on a construit des robots pour aller sur la lune mais le mettre avec des humains c'est plus difficile

L’industrie, la technique et la recherche sont déjà investies par les robots. Mais quand les considérerons-nous comme des partenaires à part entière de notre vie ? C’est à cette question que tend à répondre ce projet de desserte robotisée.

C’est à l’origine dans le domaine de l’industrie, notamment automobile, que les robots se sont d’abord et le plus durablement installés. Pour ensuite se trouver dans les zones nucléaires, les terrains minés, l’espace ou les abysses marins à très forte pression. Et c’est ainsi que le robot a su se faire une place dans ces milieux dit « extrêmes ». Mais il n’est pas encore très bien adapté au secteur du service.

Faire entrer un robot chez soi reste pour beaucoup un grand pas à franchir. Il faut donc réussir à faire rentrer le robot dans notre société de façon subtile. Alors l’acceptation du robot comme un atout pour l’être humain, et non comme un danger, deviendra possible.

A travers la desserte robotisée, nous offrons au robot une place de soutiens au service lors de cocktail. On l’intègre donc à une place de partenaire pour l’homme. Et c’est ainsi qu’il prend une place dans le champ humain sans s’y imposer de façon brusque.


\section{Le context du cocktail}

\subsection{Où se déroule un cocktail}
%lister les lieux

%tous les lieux d'évènment
%Reception officiel et ou diplomatique
%Artistique
%Culturel
%Célébration (Mariage, remise de diplome, ventes...)

%les lieux d'interconnection de modalité de transport
%Inoguration

%espace grand
%Terrasse -> la merde en faisabilité
%Hotel, Galerie, Musée
%Jardin -> critique faisabilité impossibilité
%Gare
%Aeroport
%Salle de reception
%Restaurant
%Salle de bal

\subsection{Qui sont les acteurs du cocktail}
%lister les personnes

%Prestataire* / Maitre d'Hotel / Traiteur.
%Serveurs
%Nettoyeurs
%Invités
%Organisateur (pas forcement le prop)
%Le propriétaire de la salle

\subsection{Les ambiances d'un cocktail}
%lister les qualités

%Tendu
%Affaires - diplomatique
%Classe -> ambiance feutré bien habillé.

%Détendu
%Galeire, vernissage
%Artistique -> Moins strict

%adapté à ces deux niveau visuel.

\subsection{Les complexités des niveaux d'intéractions dans le cocktail}
%lister les "`enjeux"' qui amènent les intéractions

%Pour chaque acteur il faut définir leur relation.

% Propriétaire -> Orga -> prest -> Netoyaire et Serveur(boucle) -> convives (boucles)
