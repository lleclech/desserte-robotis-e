\section{Le Software}

Après la gestion du hardware, il faut penser au support software du robot. Ce qui inclut toute la reconnaissance spatial, la gestion de groupe et la gestion de son interface tactile. IL faut donc différencier deux type de programme, ceux interne au robot et ceux externe disponible par un client.

\subsection{Programme externe au robot}

\subsubsection{Interface tactile}

Le robot doit avoir une interface tactile sur son corps de telle façon qu’il puisse renseigner les invités sur les autres invités ou les mets que le cocktail sert. C’est pourquoi il y a un développement d’interface tactile pour renseigner efficacement et rapidement les gens dans le cocktail.

\subsubsection{Prise de commande à distance}

Comme il y aura un ordinateur permettant de commander le robot à distance, il faut prévoir une interface de contrôle à distance depuis un ordinateur.

\subsection{Programme interne au robt}

\subsubsection{Reconnaissance spatial}

Pour se déplacer dans le cocktail, le robot doit savoir se repérer dans la salle. Nous pourrons donc nous placer grâce à une cartographie. Elle pourra être d’exploration SLAM.

\subsubsection{Reconnaissance de groupe}

Dans le but de bien gérer les groupes, le robot doit savoir se déplacer dans la salle et éviter les groupes sociaux. On pourra réfléchir sur les travaux de Jorge Alberto Rios Martinez et sa thèse « Socially-Aware Robot Navigation :combining Risk Assessment and Social Conventions » ~\cite{Martinez200601}.

Pour faciliter la navigation dans une foule épaisse, on pourra se baser sur les travaux d’Anne Spalanzani et Christian Laugier ou encore Victor Santos sur la navigation en population par le suivit de leader, « Following the Leader » ~\cite{followLeader}.


% Reconnaissance facial
% Reconnaissance spatial
% Reconnaissance de groupe
% Description interface tactile
